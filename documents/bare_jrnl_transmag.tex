\documentclass[12pt,a4paper]{article}
\usepackage[pdftex]{graphicx}
\usepackage[spanish]{babel}
\usepackage[utf8]{inputenc}
\usepackage[T1]{fontenc}
\usepackage{float}
\usepackage{amsmath}

\usepackage{listings}
\usepackage{xcolor}

\setlength{\parindent}{0pt} 

\lstset{
  language=R,
  basicstyle=\ttfamily\footnotesize,
  keywordstyle=\color{blue},
  commentstyle=\color{gray},
  stringstyle=\color{red},
  showstringspaces=false,
  breaklines=true
}

\hyphenation{op-tical net-works semi-conduc-tor}


\begin{document}

\title{Trabajo práctico 0: Procesamiento Digital de Imágenes}

\author{
    Diego Naranjo \\ Tecnológico de Costa Rica \\ l.naranjo.2@estudiantec.cr
    \and Gabriel Bonilla \\ Tecnológico de Costa Rica \\ g.bonilla.1@estudiantec.cr
    \and José Godínez \\ Tecnológico de Costa Rica \\ rodolfojose1996@estudiantec.cr
}

\pagestyle{myheadings}
\markboth{Instituto Tecnológico de Costa Rica, Agosto~2025}{Instituto Tecnológico de Costa Rica, Agosto~2025}


\maketitle


\section{Sistemas lineales}

Demuestre si los siguientes sistemas $L\left\{ u(t)\right\} $ (con entrada $u(t)$ y salida $g(t)=L\left\{ u(t)\right\} $, y $h(t)$ una función cualquiera) son lineales o no lineales.

Para demostrar por contraejemplo programe la función que rechace la propiedad para 100 entradas generadas aleatoriamente:

\begin{itemize}
    \item $g\left(t\right)=\int_{-\infty}^{\infty}u(t)$
    \item $g\left(t\right)=\begin{cases} 1 & u(t)>0\\ 0 & u(t)\leq0 \end{cases}\,$
    \item $g\left(t\right)=\ln\left(u(t)\right)$
    \item $g\left(t\right)=\cos\left(u(t)\right)$
    \item $g\left(t\right)=\ln\left(5^{u(t)}\right)$
\end{itemize}

\subsection{$g(t)=\int^{\infty}_{-\infty}u(t)$}

\begin{itemize}
    \item  \textbf{Homogeneidad} ($L\{\alpha u(t)\} = \alpha L\{u(t)\}$):
    \begin{align*}
        L\{u(t)\} &= \int^{\infty}_{-\infty}u(t) \\
        L\{\alpha u(t)\} &= \int^{\infty}_{-\infty}\alpha u(t) \\
        L\{\alpha u(t)\} &= \alpha \int^{\infty}_{-\infty} u(t)\\
        L\{\alpha u(t)\} &= \alpha L\{u(t)\}
    \end{align*}
    \begin{itemize}
        \item \textbf{Aditividad} ($L\{u_1(t)+u_2(t)\}=L\{u_1(t)\}+ L\{u_2(t)\}$)
        \begin{align*}
            L\{u(t)\} &= \int^{\infty}_{-\infty}u(t) \\
            L\{u_1(t)+u_2(t)\} &=  \int^{\infty}_{-\infty}u_1(t)+u_2(t)\\
            L\{u_1(t)+u_2(t)\} &=  \int^{\infty}_{-\infty}u_1(t)+ \int^{\infty}_{-\infty}u_2(t)\\
            L\{u_1(t)+u_2(t)\} &= L\{u_1(t)\} + L\{u_2(t)\}
         \end{align*}
    \end{itemize}
    \item \textbf{Conclusión:} $g(t)$ cumple las condiciones para ser un sistema lineal.
\end{itemize}

\subsection{$
\quad g(t) =
\begin{cases}
1 & u(t) > 0 \\
0 & u(t) \leq 0
\end{cases}
$}

\begin{itemize}
    \item \textbf{Homogeneidad} ($L\{\alpha u(t)\} = \alpha L\{u(t)\}$): Analizando los casos:
    \begin{itemize}
        \item caso $u(t) > 0$, $\alpha > 0$: 
        \begin{align*}
            L\{\alpha u(t)\} &=  1\\
            L\{\alpha u(t)\} &\neq \alpha L\{ u(t)\}
        \end{align*}
    \end{itemize}
    Y evaluando un contraejemplo con $u(t) = 1$ y $\alpha = 2$
    \begin{align*}
        L\{\alpha u(t)\} &\stackrel{?}{=} \alpha L\{ u(t)\}\\
        L\{2\} &\stackrel{?}{=} 2 L\{ 1\}\\
        1 &\neq 2
    \end{align*}
    Y se pueden ver con más detalle en la sección \ref{subsec:contraejemplos}
    \item \textbf{Conclusión:} $g(t)$ no cumple con las condiciones para ser un sistema lineal.

\end{itemize}

\subsection{$g(t)=\ln(u(t))$}

\begin{itemize}
    \item \textbf{Homogeneidad} ($L\{\alpha u(t)\} = \alpha L\{u(t)\}$)
    \begin{align*}
        L\{u(t)\} &= \ln(u(t))\\
        L\{\alpha u(t)\} &= \ln(\alpha u(t))\\
        L\{\alpha u(t)\} &= \ln(\alpha) + \ln(u(t))\\
        L\{\alpha u(t)\} &\neq \alpha L\{u(t)\}
    \end{align*}
    Y evaluando un contraejemplo con $u(t) = 1$ y $\alpha = 2$
    \begin{align*}
        L\{\alpha u(t)\} &\stackrel{?}{=} \alpha L\{ u(t)\}\\
        \ln(2) &\stackrel{?}{=} 2 \ln(1)\\
        0.69 &\neq 0
    \end{align*}
    Y se pueden ver con más detalle en la sección \ref{subsec:contraejemplos}
    \item \textbf{Conclusión:} $g(t)$ no cumple con las condiciones para ser un sistema lineal.
\end{itemize}

\subsection{$g(t)=\cos(u(t))$}

\begin{itemize}
    \item \textbf{Aditividad} ($L\{u_1(t)+u_2(t)\}=L\{u_1(t)\}+ L\{u_2(t)\}$)
    \begin{align*}
        L\{u(t)\} &= \cos(u(t))\\
        L\{u_1(t)+u_2(t)\} &= \cos (u_1(t)+u_2(t))\\
        \cos(a+b) &= \cos(a)\cos(b)-\sin(a)\sin(b)\\
        L\{u_1(t)+u_2(t)\} &= \cos(u_1(t))\cos(u_2(t))-\sin(u_1(t))\sin(u_2(t))\\
        L\{u_1(t)+u_2(t)\} &\neq L\{u_1(t)\}+L\{u_2(t)\} 
    \end{align*}
    Y verificando lo obtenido con un contrajemplo:
    \begin{align*}
        L\{u_1(t)+u_2(t)\} &\stackrel{?}{=} L\{u_1(t)\}+ L\{u_2(t)\}\\
        \cos(1+2) &\stackrel{?}{=} \cos(1) + \cos(2)\\
         -0.99 &\neq 0.54 + -0.41
    \end{align*}
    Y se pueden ver con más detalle en la sección \ref{subsec:contraejemplos}
    \item \textbf{Conclusión}: $g(t)$ no cumple con los criterios para ser un sistema lineal.
\end{itemize}

\subsection{$g(t)=\ln(5^{u(t)})$}

\begin{itemize}
    \item \textbf{Homogeneidad} ($L\{\alpha u(t)\} = \alpha L\{u(t)\}$)
    \begin{align*}
        g(t)&=\ln(5^{u(t)})\\
        g(t) &= u(t)\ln(5), \text{con} \ln(a^b)=b\ln(a)\\
        L\{u(t)\} &= u(t)\ln(5)\\
        L\{\alpha u(t)\} &= \alpha u(t) \ln(5)\\
        L\{\alpha u(t)\} &= \alpha L\{u(t)\}
    \end{align*}
    \item \textbf{Aditividad} ($L\{u_1(t)+u_2(t)\}=L\{u_1(t)\}+ L\{u_2(t)\}$)
        \begin{align*}
        L\{u(t)\} &= u(t)\ln(5)\\
        L\{u_1(t)+u_2(t)\} &= (u_1(t)+u_2(t)) \ln(5)\\
        L\{u_1(t)+u_2(t)\} &= \ln(5)u_1(t)+\ln(5)u_2(t)\\
        L\{u_1(t)+u_2(t)\} &= L\{u_1(t)\}+L\{u_2(t)\}
    \end{align*}
    \item \textbf{Conclusión}: $g(t)$ sí cumple con los criterios para ser un sistema lineal.
\end{itemize}

\subsection{Verificaciones de contraejemplos}
\label{subsec:contraejemplos}

Para cada uno de los sistemas que se determinaron como no lineales (b, c y d), se realizó:
\begin{itemize}
    \item Generación de conjuntos $u_1$, $u_2$ con tamaño 100 cada uno y un escalar $\alpha$. Para la función del $\ln$, se acotaron los valores menores a 0 como $0.00001$ por terminos de indefinición.
    \item Vectorialmente con torch se revisó la aditividad $L\{u_1(t)+u_2(t)\} \overset{?}{=} L\{u_1(t)\}+L\{u_2(t)\}$.
    \item Vectorialmente se revisó la homogeneidad $L\{\alpha u(t)\} = \alpha L\{u(t)\}$
\end{itemize}

Los resultados obtenidos en la figura \ref{fig:counterexamples}

\begin{figure*}[h!]
    \centering
    \includegraphics[width=\textwidth]{../img/counterexamples.PNG}
    \caption{Resultados de la prueba automatizada de búsqueda de contraejemplos en el análisis de los posibles sistemas lineales, donde los sistemas b, c y d se les encontró que no cumplen lo necesario para ser considerados lineales.}
    \label{fig:counterexamples}
\end{figure*}

\section{Interpolación bilineal}

\subsection{Proceso matemático}

Para interpolar bilinealmente el valor de la coordenada $z$ para cada uno de los puntos de una submatriz, se requiere identificar un plano que pasa por 3 puntos ya conocidos, en este caso de las esquinas. Para ello, tenemos que recordar que un plano está definido por la ecuación $ax + by + cz + d = 0$, siendo $a$, $b$ y $c$ los coeficientes del plano, $d$ su offset y $x$, $y$ y $z$ las coordenadas de un punto en $R^3$. \\

Dicho esto, para calcular la ecuación del plano en cada una de las submatrices, vamos tomar 3 de sus puntos conocidos. Los vamos a denominar $P_1$, $P_2$ y $P_3$. Con ellos, vamos a calcular 2 vectores directores plano:

\[
\vec{(P_1 - P_2)} = [x_1 - x_2, y_1 - y_2, z_1 - z_2]
\]
\[
\vec{(P_2 - P_3)} = [x_2 - x_3, y_2 - y_3, z_2 - z_3]
\]

Al estos vectores estar en el plano, sirven para obtener un vector normal del mismo, y así mismo, sus coeficientes. Esto se hace mediante el producto cruz de los vectores:

\[
\vec{(P_1 - P_2)} \times \vec{(P_2 - P_3)} = \vec{n}
\]

Una vez obtenido el vector normal $\vec{n}$, podemos definir la exuación del plano de la siguiente manera:

\[
n_1x + n_2y + n_3z + d = 0
\]

Para determinar el valor de $d$, simplemente se sustituyen, en la ecuación, los valores de $x$, $y$ y $z$ por las coordenadas de un punto conocido del plano y se despeja $d$. En este caso, el punto escogido puede ser cualquiera de $P_1$, $P_2$ y $P_3$:

\[
d = -(n_1P_{1_x} + n_2P_{1_y} + n_3P_{1_z})
\]

Ya se tienen todos los valores necesarios para poder despejar y encontrar la coordenada $z$ de un punto utilizando la ecuación del plano, en casos que se conozcan sus coordenadas $x$, $y$: 

\[
z = \frac{(-d - n_1x - n_2y)}{n_3}
\]

\subsection{Ejemplo}

La siguiente matriz $U$ representa una submatriz de una imagen a la que se le aplicó un aumento con $\alpha = 2$:

\[
U =
    \begin{bmatrix}
    1 & ? & 2 \\
    ? & ? & ? \\
    3 & ? & 4
    \end{bmatrix}
\]

Dada la submatriz $U$, se tienen los puntos conocidos:

\[
P_1 = [0,0,1], P_2 [2,0,3], P_3 = [2,2,4]
\]

Se definen los 2 vectores directores como:

\[
\vec{P_1 - P_2} = [-2,0-2], \vec{P_2 - P_3} = [2,2,4]
\]

Se calcula el vector normal:

\[
\vec{P_1 - P_2} \times \vec{P_2 - P_3} = [-4,-2,4]
\]

Se utiliza $P_1$ para obtener el valor de $d$:

\[
d = -((-4 \cdot 0) + (-2\cdot 0) + (4\cdot 1)) = -4
\]

Por último, se calcula la coordenada $z$ para cada uno de los valores faltantes de la submatriz:

\begin{align*}
\textit{Para } (0,1,z), \; & z = \frac{(4 + (4 \cdot 0) + (2 \cdot 1))}{4} = 1.5 \\[6pt]
\textit{Para } (1,0,z), \; & z = \frac{(4 + (4 \cdot 1) + (2 \cdot 0))}{4} = 2 \\[6pt]
\textit{Para } (1,1,z), \; & z = \frac{(4 + (4 \cdot 1) + (2 \cdot 1))}{4} = 2.5 \\[6pt]
\textit{Para } (1,2,z), \; & z = \frac{(4 + (4 \cdot 1) + (2 \cdot 2))}{4} = 3 \\[6pt]
\textit{Para } (2,1,z), \; & z = \frac{(4 + (4 \cdot 2) + (2 \cdot 1))}{4} = 3.5
\end{align*}

La submatriz $U$ quedaría de la siguiente manera:

\[
U =
    \begin{bmatrix}
    1 & 1.5 & 2 \\
    2 & 2.5 & 3 \\
    3 & 3.5 & 4
    \end{bmatrix}
\]

\section{Implementación del algoritmo Bilinear para el aumentado de tamaño de una imagen}

\subsection{Estimación del punto P}

\subsection{Implementación para cambiar el tamaño de una imágen}

\subsection{Benchmark para cuantificar el error en el cambio de tamaño bilineal}

\subsubsection{Definición del conjunto de imágenes}
Se seleccionó un conjunto de 10 imágenes variadas para poner a prueba 
el algoritmo de cambio de tamaño bilineal en distintos escenarios. 
Estas imágenes permiten evaluar el rendimiento del método en contextos diversos.

\subsubsection{Proceso experimental}
El flujo experimental consistió en los siguientes pasos:
\begin{enumerate}
    \item Cargar cada imagen y normalizarla al rango $[0,1]$.
    \item Redimensionar la imagen original a un tamaño reducido (ej. mitad de la resolución).
    \item Volver a expandir la imagen reducida al tamaño original utilizando interpolación bilineal.
    \item Calcular métricas de error comparando la imagen original y la reconstruida.
\end{enumerate}

Además, se implementaron las métricas \textbf{PSNR} y una métrica de 
\textbf{similitud porcentual} derivada del MSE, con el fin de contar 
con medidas tanto absolutas como relativas de calidad.

\subsubsection{Métricas de evaluación}
Se utilizaron las siguientes métricas:
\begin{itemize}
    \item \textbf{MSE (Mean Squared Error):} mide el error promedio al cuadrado 
    entre la imagen original y la reconstruida. Valores bajos indican mayor similitud.
    \item \textbf{PSNR (Peak Signal-to-Noise Ratio):} mide la relación señal-ruido 
    en decibelios. Valores altos indican mejor calidad.
    \item \textbf{Similitud porcentual:} métrica definida en el rango $[0,100]$, 
    donde $100$ representa similitud máxima.
\end{itemize}

\subsubsection{Resultados del benchmark}
La tabla \ref{tab:bilinear-benchmark} muestra los resultados obtenidos 
para las 15 imágenes procesadas en el experimento y el factor de cambio aplicado 
(escalado a la mitad y posterior reescalado al tamaño original).

\begin{table}[H]
\centering
\caption{Resultados del benchmark de interpolación bilineal.}
\label{tab:bilinear-benchmark}
\begin{tabular}{c|c|c|c}
\hline
Imagen & MSE & PSNR (dB) & Similitud (\%) \\ \hline
0  & 0.001813 & 27.416 & 99.8187 \\
1  & 0.001829 & 27.377 & 99.8171 \\
2  & 0.001506 & 28.221 & 99.8494 \\
3  & 0.000733 & 31.348 & 99.9267 \\
4  & 0.000872 & 30.597 & 99.9128 \\
5  & 0.002168 & 26.640 & 99.7832 \\
6  & 0.000563 & 32.493 & 99.9437 \\
7  & 0.000555 & 32.559 & 99.9445 \\
8  & 0.000550 & 32.597 & 99.9450 \\
9  & 0.002302 & 26.379 & 99.7698 \\
10 & 0.001837 & 27.359 & 99.8163 \\
11 & 0.000502 & 32.991 & 99.9498 \\
12 & 0.001599 & 27.963 & 99.8401 \\
13 & 0.001257 & 29.005 & 99.8743 \\
14 & 0.001680 & 27.746 & 99.8320 \\ \hline
\textbf{Media} & 0.001318 & 29.379 & 99.8682 \\
\textbf{Desv.~Est.} & 0.000636 & 2.437 & 0.0636 \\ \hline
\end{tabular}
\end{table}

\subsubsection{Discusión de resultados}
Se observa que el error cuadrático medio (MSE) se mantiene en el orden de $10^{-3}$, lo cual indica que la interpolación bilineal preserva la información de manera aceptable. Las métricas PSNR se encuentran en un rango de $26$ a $33$ dB, lo cual es consistente con una calidad percibida adecuada en aplicaciones de visión por computadora.  

En cuanto a la similitud porcentual, todos los valores superan el $99.7\%$, lo cual refuerza la idea de que el cambio de tamaño bilineal introduce un error bajo. La desviación estándar reportada es pequeña, lo cual demuestra estabilidad del método frente a diferentes imágenes.

En conclusión, el algoritmo de cambio de tamaño bilineal presenta un desempeño sólido en la preservación de calidad visual, aunque se aprecia que en imágenes con más detalle fino (ej. índices 5 y 9) el error tiende a ser mayor debido a la pérdida de información en el proceso de reducción.

\end{document}



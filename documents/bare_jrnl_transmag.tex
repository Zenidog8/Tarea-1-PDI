\documentclass[12pt,a4paper]{article}
\usepackage[pdftex]{graphicx}
\usepackage[spanish]{babel}
\usepackage[utf8]{inputenc}
\usepackage[T1]{fontenc}

\usepackage{amsmath}

\usepackage{listings}
\usepackage{xcolor}

\setlength{\parindent}{0pt} 

\lstset{
  language=R,
  basicstyle=\ttfamily\footnotesize,
  keywordstyle=\color{blue},
  commentstyle=\color{gray},
  stringstyle=\color{red},
  showstringspaces=false,
  breaklines=true
}

\hyphenation{op-tical net-works semi-conduc-tor}


\begin{document}

\title{Trabajo práctico 0: Procesamiento Digital de Imágenes}

\author{
    Diego Naranjo \\ Tecnológico de Costa Rica \\ l.naranjo.2@estudiantec.cr
    \and Gabriel Bonilla \\ Tecnológico de Costa Rica \\ g.bonilla.1@estudiantec.cr
    \and José Godínez \\ Tecnológico de Costa Rica \\ rodolfojose1996@estudiantec.cr
}


\markboth{Instituto Tecnológico de Costa Rica, Agosto~2025}%

\maketitle


\section{Interpolación bilineal}

\subsection{Proceso matemático}

Para interpolar bilinealmente el valor de la coordenada $z$ para cada uno de los puntos de una submatriz, se requiere identificar un plano que pasa por 3 puntos ya conocidos, en este caso de las esquinas. Para ello, tenemos que recordar que un plano está definido por la ecuación $ax + by + cz + d = 0$, siendo $a$, $b$ y $c$ los coeficientes del plano, $d$ su offset y $x$, $y$ y $z$ las coordenadas de un punto en $R^3$. \\

Dicho esto, para calcular la ecuación del plano en cada una de las submatrices, vamos tomar 3 de sus puntos conocidos. Los vamos a denominar $P_1$, $P_2$ y $P_3$. Con ellos, vamos a calcular 2 vectores directores plano:

\[
\vec{(P_1 - P_2)} = [x_1 - x_2, y_1 - y_2, z_1 - z_2]
\]
\[
\vec{(P_2 - P_3)} = [x_2 - x_3, y_2 - y_3, z_2 - z_3]
\]

Al estos vectores estar en el plano, sirven para obtener un vector normal del mismo, y así mismo, sus coeficientes. Esto se hace mediante el producto cruz de los vectores:

\[
\vec{(P_1 - P_2)} \times \vec{(P_2 - P_3)} = \vec{n}
\]

Una vez obtenido el vector normal $\vec{n}$, podemos definir la exuación del plano de la siguiente manera:

\[
n_1x + n_2y + n_3z + d = 0
\]

Para determinar el valor de $d$, simplemente se sustituyen, en la ecuación, los valores de $x$, $y$ y $z$ por las coordenadas de un punto conocido del plano y se despeja $d$. En este caso, el punto escogido puede ser cualquiera de $P_1$, $P_2$ y $P_3$:

\[
d = -(n_1P_{1_x} + n_2P_{1_y} + n_3P_{1_z})
\]

Ya se tienen todos los valores necesarios para poder despejar y encontrar la coordenada $z$ de un punto utilizando la ecuación del plano, en casos que se conozcan sus coordenadas $x$, $y$: 

\[
z = \frac{(-d - n_1x - n_2y)}{n_3}
\]

\subsection{Ejemplo}

La siguiente matriz $U$ representa una submatriz de una imagen a la que se le aplicó un aumento con $\alpha = 2$:

\[
U =
    \begin{bmatrix}
    1 & ? & 2 \\
    ? & ? & ? \\
    3 & ? & 4
    \end{bmatrix}
\]

Dada la submatriz $U$, se tienen los puntos conocidos:

\[
P_1 = [0,0,1], P_2 [2,0,3], P_3 = [2,2,4]
\]

Se definen los 2 vectores directores como:

\[
\vec{P_1 - P_2} = [-2,0-2], \vec{P_2 - P_3} = [2,2,4]
\]

Se calcula el vector normal:

\[
\vec{P_1 - P_2} \times \vec{P_2 - P_3} = [-4,-2,4]
\]

Se utiliza $P_1$ para obtener el valor de $d$:

\[
d = -((-4 \cdot 0) + (-2\cdot 0) + (4\cdot 1)) = -4
\]

Por último, se calcula la coordenada $z$ para cada uno de los valores faltantes de la submatriz:

\begin{align*}
\textit{Para } (0,1,z), \; & z = \frac{(4 + (4 \cdot 0) + (2 \cdot 1))}{4} = 1.5 \\[6pt]
\textit{Para } (1,0,z), \; & z = \frac{(4 + (4 \cdot 1) + (2 \cdot 0))}{4} = 2 \\[6pt]
\textit{Para } (1,1,z), \; & z = \frac{(4 + (4 \cdot 1) + (2 \cdot 1))}{4} = 2.5 \\[6pt]
\textit{Para } (1,2,z), \; & z = \frac{(4 + (4 \cdot 1) + (2 \cdot 2))}{4} = 3 \\[6pt]
\textit{Para } (2,1,z), \; & z = \frac{(4 + (4 \cdot 2) + (2 \cdot 1))}{4} = 3.5
\end{align*}

La submatriz $U$ quedaría de la siguiente manera:

\[
U =
    \begin{bmatrix}
    1 & 1.5 & 2 \\
    2 & 2.5 & 3 \\
    3 & 3.5 & 4
    \end{bmatrix}
\]


\end{document}


